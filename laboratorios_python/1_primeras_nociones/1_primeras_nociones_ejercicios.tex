\documentclass[a4paper,spanish,12pt]{article}
\usepackage{graphicx,amscd,color,amsmath,amsfonts,amssymb,amsthm}
\usepackage{babel}
\usepackage[utf8]{inputenc}
\usepackage{eurosym}

%%% retoques a dimensiones
\setlength{\topmargin}{-1cm}
\addtolength{\evensidemargin}{-2cm}
\addtolength{\oddsidemargin}{-2cm}
\addtolength{\textwidth}{3.5cm}

%%%%%% conjuntos de numeros %%%%%%%%
\newcommand{\R}{{\mathbb R}}
\newcommand{\ds}{{\displaystyle}}

\begin{document}

\pagestyle{empty}

\centerline{\huge\scshape Laboratorio 1: Soluciones de los ejercicios}
\vspace{25pt}

En este documento se recopilan soluciones orientativas a los ejercicios
del Laboratorio 1. Se muestran tanto las expresiones matemáticas como
ejemplos de código en Python/NumPy.

%%%%%%%%%%%%%%%%%%%%%%%%%%%%%%%%%%%%%%%%%%%%%%%%%%%%%%%%%%%%%%%%%%%%%%
\section*{1. Vectores y Matrices}

\subsection*{Ejercicio 1}

\textbf{Enunciado.} Definir en Python (usando NumPy) las matrices y vectores
\[
A=\begin{pmatrix}
-1 & \sqrt{5} & 2 & \pi \\
0 & 3 & -1 & 4 \\
\pi & \sqrt[3]{5} & 0 & 1
\end{pmatrix},\quad
B=\begin{pmatrix}
3 & \frac{2}{5} & 0 \\
0 & -1 & 4 \\
\pi & 0 & 1
\end{pmatrix},
\]
\[
v=\begin{pmatrix}
4 \\ -3 \\ 2
\end{pmatrix},\quad
w=\begin{pmatrix}
\sqrt{7} \\ 0 \\ 2 \\ \pi
\end{pmatrix}.
\]

\textbf{Solución (código en Python).}
\begin{verbatim}
import numpy as np

A = np.array([[-1, np.sqrt(5), 2, np.pi],
              [ 0, 3,         -1, 4],
              [ np.pi, 5**(1/3), 0, 1]])

B = np.array([[3, 2/5, 0],
              [0, -1, 4],
              [np.pi, 0, 1]])

v = np.array([[4],
              [-3],
              [2]])

w = np.array([[np.sqrt(7)],
              [0],
              [2],
              [np.pi]])
\end{verbatim}

\subsection*{Ejercicio 2}

\textbf{Enunciado.} Realizar todos los productos matriz--matriz y
matriz--vector que las dimensiones admitan.

\textbf{Solución.} Con las dimensiones indicadas,
\[
A\in\R^{3\times 4},\quad
B\in\R^{3\times 3},\quad
v\in\R^{3\times 1},\quad
w\in\R^{4\times 1}.
\]
Los productos válidos (del tipo matriz--matriz o matriz--vector) son:
\[
B\,A,\quad B\,v,\quad A\,w.
\]

En Python (usando \verb|@| para el producto matricial):
\begin{verbatim}
BA = B @ A   # 3x3 por 3x4 -> 3x4
Bv = B @ v   # 3x3 por 3x1 -> 3x1
Aw = A @ w   # 3x4 por 4x1 -> 3x1
\end{verbatim}

%%%%%%%%%%%%%%%%%%%%%%%%%%%%%%%%%%%%%%%%%%%%%%%%%%%%%%%%%%%%%%%%%%%%%%
\section*{2. Sucesiones}

\textbf{Ejercicio.} Consideramos la sucesión
\[
2, 5, 8, 11, 14, 17, 20, 23, 26.
\]
Definir esta sucesión de tres formas distintas y guardarlas bajo sendas
variables \verb|v1|, \verb|v2| y \verb|v3|.

\textbf{Solución (código en Python).}
\begin{verbatim}
import numpy as np

# Forma 1: arange con paso 3 (hasta 27 excluido)
v1 = np.arange(2, 27, 3)

# Forma 2: 2 + 3*k, k = 0,...,8
v2 = 2 + 3 * np.arange(9)

# Forma 3: linspace indicando número de puntos
v3 = np.linspace(2, 26, 9)  # 9 puntos entre 2 y 26
\end{verbatim}

%%%%%%%%%%%%%%%%%%%%%%%%%%%%%%%%%%%%%%%%%%%%%%%%%%%%%%%%%%%%%%%%%%%%%%
\section*{3. Funciones matemáticas y álgebra lineal}

\subsection*{Ejercicio 1}

\textbf{Enunciado.} Resolver el sistema, planteándolo de forma matricial:
\[
\begin{aligned}
3x + 2y        &= 0, \\
2x     -2z + t &= 1, \\
\phantom{2x+}y +4z -3t &= -2, \\
x +5y - z -3t &= 4.
\end{aligned}
\]

\textbf{Solución.} Escribimos el sistema como $A\mathbf{x} = \mathbf{b}$,
donde
\[
A =
\begin{pmatrix}
3 & 2 & 0 & 0 \\
2 & 0 & -2 & 1 \\
0 & 1 & 4 & -3 \\
1 & 5 & -1 & -3
\end{pmatrix},\quad
\mathbf{x} =
\begin{pmatrix} x \\ y \\ z \\ t \end{pmatrix},\quad
\mathbf{b} =
\begin{pmatrix} 0 \\ 1 \\ -2 \\ 4 \end{pmatrix}.
\]

En Python:
\begin{verbatim}
import numpy as np

A = np.array([[3, 2,  0,  0],
              [2, 0, -2,  1],
              [0, 1,  4, -3],
              [1, 5, -1, -3]])

b = np.array([0, 1, -2, 4])

x_sol = np.linalg.solve(A, b)
print(x_sol)
\end{verbatim}

El resultado es
\[
x = -\frac{14}{65},\quad
y = \frac{21}{65},\quad
z = -\frac{64}{65},\quad
t = -\frac{7}{13}.
\]

\subsection*{Ejercicio 2}

\textbf{Enunciado.} Dados los vectores
\[
x=\begin{pmatrix}1 \\2\\-3\\5\end{pmatrix},\quad
y=\begin{pmatrix}4 \\-4\\1\\2\end{pmatrix},\quad
z=\begin{pmatrix}3 \\1\\2\\-4\end{pmatrix},
\]
formar una matriz cuyas dos primeras columnas sean todo ceros, sus siguientes tres
columnas sean los vectores $x, y, z$ y sus últimas columnas sean todo unos.

\textbf{Solución.} Una posibilidad es tomar dos columnas de unos, de forma que
la matriz total tenga $2 + 3 + 2 = 7$ columnas. Simbólicamente:
\[
M = \begin{pmatrix}
0 & 0 & 1 & 4 & 3 & 1 & 1 \\
0 & 0 & 2 & -4 & 1 & 1 & 1 \\
0 & 0 & -3 & 1 & 2 & 1 & 1 \\
0 & 0 & 5 & 2 & -4 & 1 & 1
\end{pmatrix}.
\]

En Python:
\begin{verbatim}
import numpy as np

x = np.array([[ 1],
              [ 2],
              [-3],
              [ 5]])

y = np.array([[ 4],
              [-4],
              [ 1],
              [ 2]])

z = np.array([[ 3],
              [ 1],
              [ 2],
              [-4]])

zeros_cols = np.zeros((4, 2))
ones_cols  = np.ones((4, 2))

M = np.hstack([zeros_cols, x, y, z, ones_cols])
\end{verbatim}

\subsection*{Ejercicio 3}

\textbf{Enunciado.} En la matriz del apartado anterior, acceder a la
submatriz formada por las filas 1, 3 y 4 y las columnas 1, 2 y 6
(teniendo en cuenta que en Python los índices empiezan en 0).
Calcular:
\begin{itemize}
  \item el tamaño de esa submatriz (número de filas y columnas),
  \item la suma de todos sus elementos.
\end{itemize}

\textbf{Solución.} Con la matriz $M$ anterior,
las filas 1, 3 y 4 en numeración matemática corresponden a los
índices 0, 2 y 3 de Python; las columnas 1, 2 y 6 corresponden a los
índices 0, 1 y 5. Obtenemos la submatriz
\[
S = M_{\{1,3,4\},\{1,2,6\}} =
\begin{pmatrix}
0 & 0 & 1 \\
0 & 0 & 1 \\
0 & 0 & 1
\end{pmatrix}.
\]
Por tanto, el tamaño de $S$ es $3\times 3$ y la suma de sus elementos es
\[
0+0+1+0+0+1+0+0+1 = 3.
\]

En Python:
\begin{verbatim}
filas = [0, 2, 3]      # filas 1, 3, 4 (notación matemática)
columnas = [0, 1, 5]   # columnas 1, 2, 6

S = M[np.ix_(filas, columnas)]
print("Submatriz S =\n", S)
print("Tamaño de S:", S.shape)
print("Suma de los elementos de S:", np.sum(S))
\end{verbatim}

%%%%%%%%%%%%%%%%%%%%%%%%%%%%%%%%%%%%%%%%%%%%%%%%%%%%%%%%%%%%%%%%%%%%%%
\section*{4. Funciones definidas por el usuario}

\subsection*{Ejercicio 1}

\textbf{Enunciado.} Crear una función (en un archivo \verb|.py|) cuyos
parámetros de entrada sean una matriz invertible \verb|A| y un vector \verb|b|,
y que devuelva la solución del sistema $Ax = b$ usando
\verb|np.linalg.solve|.

\textbf{Solución (código en Python).}
\begin{verbatim}
import numpy as np

def resolver_sistema(A, b):
    """Resuelve el sistema lineal A x = b."""
    x = np.linalg.solve(A, b)
    return x

# Ejemplo de uso:
# A = np.array([[1, 2], [3, 4]])
# b = np.array([5, 6])
# sol = resolver_sistema(A, b)
# print(sol)
\end{verbatim}

%%%%%%%%%%%%%%%%%%%%%%%%%%%%%%%%%%%%%%%%%%%%%%%%%%%%%%%%%%%%%%%%%%%%%%
\subsection*{Ejercicio 2}

\textbf{Enunciado.} Crear una función cuyos parámetros de entrada sean una
matriz, un vector con tantas coordenadas como filas de la matriz y un número no
mayor que el número de columnas de la matriz, y que devuelva la matriz
sustituyendo la columna indicada por el número por el vector (utilizando
indexado y asignación en NumPy).

\textbf{Solución (código en Python).}
\begin{verbatim}
import numpy as np

def sustituir_columna(A, v, k):
    """
    Sustituye la columna k de la matriz A por el vector v.
    Se asume 0 <= k < número de columnas de A
    y que v tiene el mismo número de filas que A.
    """
    B = A.copy()
    B[:, k] = v
    return B

# Ejemplo de uso:
# A = np.array([[1, 2, 3],
#               [4, 5, 6]])
# v = np.array([10, 20])
# B = sustituir_columna(A, v, 1)  # reemplaza la columna 1
# print(B)
\end{verbatim}

%%%%%%%%%%%%%%%%%%%%%%%%%%%%%%%%%%%%%%%%%%%%%%%%%%%%%%%%%%%%%%%%%%%%%%
\subsection*{Ejercicio 3}

\textbf{Enunciado.} Crear una función cuyos parámetros sean una matriz
y que devuelva:
\begin{itemize}
  \item la suma de todos sus elementos,
  \item la suma de los elementos de la diagonal principal (traza).
\end{itemize}

\textbf{Solución (código en Python).}
\begin{verbatim}
import numpy as np

def suma_y_traza(A):
    """
    Devuelve la suma de todos los elementos de A
    y la suma de los elementos de la diagonal principal (traza).
    """
    suma_total = np.sum(A)
    traza = np.trace(A)
    return suma_total, traza

# Ejemplo de uso:
# A = np.array([[1, 2, 3],
#               [4, 5, 6],
#               [7, 8, 9]])
# s, t = suma_y_traza(A)
# print("Suma total:", s)   # 45
# print("Traza:", t)        # 1 + 5 + 9 = 15
\end{verbatim}

%%%%%%%%%%%%%%%%%%%%%%%%%%%%%%%%%%%%%%%%%%%%%%%%%%%%%%%%%%%%%%%%%%%%%%
\subsection*{Ejercicio 4}

\textbf{Enunciado.} Crear una función que reciba como entrada un array de NumPy
y devuelva dos valores: la suma de sus elementos y el valor medio (media
aritmética). Probarla con un vector de 5 números.

\textbf{Solución (código en Python).}
\begin{verbatim}
import numpy as np

def suma_y_media(x):
    """
    Devuelve la suma de los elementos de x y su media aritmética.
    """
    suma = np.sum(x)
    media = suma / x.size
    return suma, media

# Ejemplo de uso:
# v = np.array([1, 2, 3, 4, 5])
# s, m = suma_y_media(v)
# print("Suma:", s)    # 15
# print("Media:", m)   # 3.0
\end{verbatim}

%%%%%%%%%%%%%%%%%%%%%%%%%%%%%%%%%%%%%%%%%%%%%%%%%%%%%%%%%%%%%%%%%%%%%%
\subsection*{Ejercicio 5}

\textbf{Enunciado.} Crear una función que reciba como entrada un array
bidimensional \verb|A| y un número \verb|k| y devuelva la suma de los elementos
de la fila \verb|k| y la suma de los elementos de la columna \verb|k|.
Probarla con una matriz $3\times 3$ sencilla.

\textbf{Solución (código en Python).}
\begin{verbatim}
import numpy as np

def suma_fila_columna(A, k):
    """
    Devuelve la suma de la fila k y la suma de la columna k de A.
    Se asume 0 <= k < número de filas/columnas.
    """
    suma_fila = np.sum(A[k, :])
    suma_col  = np.sum(A[:, k])
    return suma_fila, suma_col

# Ejemplo de uso:
# A = np.array([[1, 2, 3],
#               [4, 5, 6],
#               [7, 8, 9]])
# sf, sc = suma_fila_columna(A, 1)  # fila y columna de índice 1
# print("Suma fila 1:", sf)   # 2 + 5 + 8 = 15
# print("Suma columna 1:", sc) # 4 + 5 + 6 = 15
\end{verbatim}

%%%%%%%%%%%%%%%%%%%%%%%%%%%%%%%%%%%%%%%%%%%%%%%%%%%%%%%%%%%%%%%%%%%%%%

\end{document}

