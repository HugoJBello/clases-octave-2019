\documentclass[a4paper,11pt]{article}

\usepackage[spanish]{babel}
\usepackage[utf8]{inputenc}
\usepackage[T1]{fontenc}
\usepackage{hyperref}
\usepackage{geometry}
\usepackage{enumitem}
\geometry{margin=2.5cm}
\setlist[itemize]{noitemsep,topsep=2pt}
\setlist[enumerate]{noitemsep,topsep=2pt}

\title{Guía rápida para instalar Visual Studio Code y Python}
\author{Grado en \dots}
\date{}

\begin{document}
\maketitle

\section*{Objetivo}
Al final de esta guía el alumnado deberá ser capaz de:
\begin{itemize}
  \item Tener instalado Visual Studio Code (VS~Code).
  \item Tener instalado Python 3 y \texttt{pip}.
  \item Usar Python y \texttt{pip} desde la terminal integrada de VS~Code.
\end{itemize}

Las instrucciones están pensadas para las versiones actuales de Windows, macOS y Linux, con especial énfasis en Windows.

\section*{1. Instalar Visual Studio Code}

\subsection*{1.1. Windows (recomendado)}
\begin{enumerate}
  \item Abrir el navegador y entrar en: \href{https://code.visualstudio.com}{https://code.visualstudio.com}.
  \item Pulsar en \textbf{Download for Windows} y descargar el instalador.
  \item Ejecutar el instalador (\texttt{VSCodeSetup\_x64.exe}) y aceptar la licencia.
  \item Opcional pero recomendable: marcar las casillas que integran VS~Code en el menú contextual (\emph{``Open with Code''}).
  \item Finalizar el asistente dejando la opción de actualizaciones automáticas activada.
\end{enumerate}

\subsection*{1.2. macOS}
\begin{enumerate}
  \item Ir a \href{https://code.visualstudio.com}{https://code.visualstudio.com} y descargar la versión para macOS.
  \item Abrir el fichero \texttt{.zip} descargado y arrastrar \texttt{Visual Studio Code.app} a \texttt{/Applications}.
  \item (Opcional) Desde VS~Code, abrir la Paleta de Comandos (\texttt{Cmd+Shift+P}), escribir \texttt{shell command} y elegir \emph{``Install 'code' command in PATH''} para poder lanzar VS~Code desde la terminal con \texttt{code}.
\end{enumerate}

\subsection*{1.3. Linux (Ubuntu/Debian como ejemplo)}
\begin{itemize}
  \item Opción sencilla: desde \href{https://code.visualstudio.com}{https://code.visualstudio.com} descargar el paquete \texttt{.deb} (para Debian/Ubuntu) o \texttt{.rpm} (para Fedora, etc.) y abrirlo con el gestor de paquetes.
  \item En Ubuntu/Debian, una vez descargado el \texttt{.deb}:
  \begin{enumerate}
    \item Abrir una terminal.
    \item Ejecutar:
\begin{verbatim}
sudo apt install ./code_*_amd64.deb
\end{verbatim}
  \end{enumerate}
\end{itemize}

\section*{2. Instalar Python 3 y pip}

\subsection*{2.1. Windows (muy importante: añadir al PATH)}
\begin{enumerate}
  \item Entrar en la web oficial de Python: \href{https://www.python.org/downloads/}{https://www.python.org/downloads/}.
  \item Pulsar \textbf{Download Python 3.x.x} (la versión estable más reciente).
  \item Ejecutar el instalador descargado (\texttt{python-3.x.x-amd64.exe}).
  \item \textbf{Muy importante:} en la primera pantalla marcar la casilla:
  \begin{center}
    \textbf{\texttt{Add python.exe to PATH}}
  \end{center}
  \item Pulsar \textbf{Customize installation} o \textbf{Install Now} (cualquiera de las dos deja Python en el PATH si se marcó la casilla anterior).
  \item Dejar marcada la opción de instalar \texttt{pip} (suele venir activada por defecto).
  \item Finalizar la instalación.
\end{enumerate}

\paragraph{Comprobación en VS~Code (Windows).}
\begin{enumerate}
  \item Abrir VS~Code.
  \item Menú \textbf{Terminal} $\rightarrow$ \textbf{New Terminal}.
  \item En la terminal que aparece abajo, escribir:
\begin{verbatim}
python --version
pip --version
\end{verbatim}
  \item Si todo está correcto, deberán aparecer la versión de Python y de \texttt{pip}. 
  \item Si \texttt{python} no se reconoce, probar:
\begin{verbatim}
py -3 --version
\end{verbatim}
  \item Si tampoco funciona, repetir la instalación de Python asegurándose de marcar \emph{``Add python.exe to PATH''}.
\end{enumerate}

\subsection*{2.2. macOS}
\begin{enumerate}
  \item Ir a \href{https://www.python.org/downloads/}{https://www.python.org/downloads/} y descargar el instalador para macOS (\texttt{.pkg}).
  \item Ejecutar el \texttt{.pkg} y seguir los pasos del asistente.
  \item Abrir la aplicación \texttt{Terminal} y comprobar:
\begin{verbatim}
python3 --version
pip3 --version
\end{verbatim}
  \item En VS~Code, abrir una terminal (\texttt{Terminal $\rightarrow$ New Terminal}) y usar \texttt{python3} y \texttt{pip3} en lugar de \texttt{python} y \texttt{pip} si es necesario.
\end{enumerate}

\subsection*{2.3. Linux (Ubuntu/Debian)}
En muchas distribuciones Python 3 ya viene instalado. Para asegurarlo (y añadir \texttt{pip}):
\begin{verbatim}
sudo apt update
sudo apt install python3 python3-pip
\end{verbatim}
Después, en VS~Code, en la terminal integrada:
\begin{verbatim}
python3 --version
pip3 --version
\end{verbatim}

\section*{3. Configurar el soporte de Python en VS~Code}

\subsection*{3.1. Extensión de Python}
\begin{enumerate}
  \item Abrir VS~Code.
  \item En la barra lateral izquierda, pulsar en el icono de extensiones (cuadro con cuatro bloques) o usar \texttt{Ctrl+Shift+X}.
  \item Buscar \texttt{Python}.
  \item Instalar la extensión oficial de Microsoft (\emph{Python} --- publisher \emph{Microsoft}).
\end{enumerate}

\subsection*{3.2. Seleccionar el intérprete de Python}
\begin{enumerate}
  \item Abrir la Paleta de Comandos: \texttt{Ctrl+Shift+P} (Windows/Linux) o \texttt{Cmd+Shift+P} (macOS).
  \item Escribir \texttt{Python: Select Interpreter}.
  \item Elegir el Python instalado (normalmente aparecerá como \texttt{Python 3.x.x} en una ruta estándar).
\end{enumerate}

\section*{4. Primera prueba: ``Hola, mundo''}

\begin{enumerate}
  \item En VS~Code, crear una carpeta para la asignatura y abrirla (\textbf{File $\rightarrow$ Open Folder}).
  \item Crear un archivo nuevo llamado \texttt{hola.py}.
  \item Escribir:
\begin{verbatim}
print("Hola, mundo")
\end{verbatim}
  \item Guardar el archivo.
  \item Abrir la terminal integrada (\textbf{Terminal $\rightarrow$ New Terminal}) y ejecutar:
\begin{verbatim}
python hola.py
\end{verbatim}
  (o \texttt{python3 hola.py} en macOS/Linux).
  \item Debería aparecer el mensaje \texttt{Hola, mundo} en la terminal. Si es así, el entorno está listo.
\end{enumerate}

\vfill
\noindent\textbf{Nota:} es recomendable que el alumnado traiga estas herramientas ya instaladas a la primera sesión de prácticas.

\end{document}
\documentclass[a4paper,11pt]{article}

\usepackage[spanish]{babel}
\usepackage[utf8]{inputenc}
\usepackage[T1]{fontenc}
\usepackage{hyperref}
\usepackage{geometry}
\usepackage{enumitem}
\geometry{margin=2.5cm}
\setlist[itemize]{noitemsep,topsep=2pt}
\setlist[enumerate]{noitemsep,topsep=2pt}

\title{Guía rápida para instalar Visual Studio Code y Python}
\author{Grado en \dots}
\date{}

\begin{document}
\maketitle

\section*{Objetivo}
Al final de esta guía el alumnado deberá ser capaz de:
\begin{itemize}
  \item Tener instalado Visual Studio Code (VS~Code).
  \item Tener instalado Python 3 y \texttt{pip}.
  \item Usar Python y \texttt{pip} desde la terminal integrada de VS~Code.
\end{itemize}

Las instrucciones están pensadas para las versiones actuales de Windows, macOS y Linux, con especial énfasis en Windows.

\section*{1. Instalar Visual Studio Code}

\subsection*{1.1. Windows (recomendado)}
\begin{enumerate}
  \item Abrir el navegador y entrar en: \href{https://code.visualstudio.com}{https://code.visualstudio.com}.
  \item Pulsar en \textbf{Download for Windows} y descargar el instalador.
  \item Ejecutar el instalador (\texttt{VSCodeSetup\_x64.exe}) y aceptar la licencia.
  \item Opcional pero recomendable: marcar las casillas que integran VS~Code en el menú contextual (\emph{``Open with Code''}).
  \item Finalizar el asistente dejando la opción de actualizaciones automáticas activada.
\end{enumerate}

\subsection*{1.2. macOS}
\begin{enumerate}
  \item Ir a \href{https://code.visualstudio.com}{https://code.visualstudio.com} y descargar la versión para macOS.
  \item Abrir el fichero \texttt{.zip} descargado y arrastrar \texttt{Visual Studio Code.app} a \texttt{/Applications}.
  \item (Opcional) Desde VS~Code, abrir la Paleta de Comandos (\texttt{Cmd+Shift+P}), escribir \texttt{shell command} y elegir \emph{``Install 'code' command in PATH''} para poder lanzar VS~Code desde la terminal con \texttt{code}.
\end{enumerate}

\subsection*{1.3. Linux (Ubuntu/Debian como ejemplo)}
\begin{itemize}
  \item Opción sencilla: desde \href{https://code.visualstudio.com}{https://code.visualstudio.com} descargar el paquete \texttt{.deb} (para Debian/Ubuntu) o \texttt{.rpm} (para Fedora, etc.) y abrirlo con el gestor de paquetes.
  \item En Ubuntu/Debian, una vez descargado el \texttt{.deb}:
  \begin{enumerate}
    \item Abrir una terminal.
    \item Ejecutar:
\begin{verbatim}
sudo apt install ./code_*_amd64.deb
\end{verbatim}
  \end{enumerate}
\end{itemize}

\section*{2. Instalar Python 3 y pip}

\subsection*{2.1. Windows (muy importante: añadir al PATH)}
\begin{enumerate}
  \item Entrar en la web oficial de Python: \href{https://www.python.org/downloads/}{https://www.python.org/downloads/}.
  \item Pulsar \textbf{Download Python 3.x.x} (la versión estable más reciente).
  \item Ejecutar el instalador descargado (\texttt{python-3.x.x-amd64.exe}).
  \item \textbf{Muy importante:} en la primera pantalla marcar la casilla:
  \begin{center}
    \textbf{\texttt{Add python.exe to PATH}}
  \end{center}
  \item Pulsar \textbf{Customize installation} o \textbf{Install Now} (cualquiera de las dos deja Python en el PATH si se marcó la casilla anterior).
  \item Dejar marcada la opción de instalar \texttt{pip} (suele venir activada por defecto).
  \item Finalizar la instalación.
\end{enumerate}

\paragraph{Comprobación en VS~Code (Windows).}
\begin{enumerate}
  \item Abrir VS~Code.
  \item Menú \textbf{Terminal} $\rightarrow$ \textbf{New Terminal}.
  \item En la terminal que aparece abajo, escribir:
\begin{verbatim}
python --version
pip --version
\end{verbatim}
  \item Si todo está correcto, deberán aparecer la versión de Python y de \texttt{pip}. 
  \item Si \texttt{python} no se reconoce, probar:
\begin{verbatim}
py -3 --version
\end{verbatim}
  \item Si tampoco funciona, repetir la instalación de Python asegurándose de marcar \emph{``Add python.exe to PATH''}.
\end{enumerate}

\subsection*{2.2. macOS}
\begin{enumerate}
  \item Ir a \href{https://www.python.org/downloads/}{https://www.python.org/downloads/} y descargar el instalador para macOS (\texttt{.pkg}).
  \item Ejecutar el \texttt{.pkg} y seguir los pasos del asistente.
  \item Abrir la aplicación \texttt{Terminal} y comprobar:
\begin{verbatim}
python3 --version
pip3 --version
\end{verbatim}
  \item En VS~Code, abrir una terminal (\texttt{Terminal $\rightarrow$ New Terminal}) y usar \texttt{python3} y \texttt{pip3} en lugar de \texttt{python} y \texttt{pip} si es necesario.
\end{enumerate}

\subsection*{2.3. Linux (Ubuntu/Debian)}
En muchas distribuciones Python 3 ya viene instalado. Para asegurarlo (y añadir \texttt{pip}):
\begin{verbatim}
sudo apt update
sudo apt install python3 python3-pip
\end{verbatim}
Después, en VS~Code, en la terminal integrada:
\begin{verbatim}
python3 --version
pip3 --version
\end{verbatim}

\section*{3. Configurar el soporte de Python en VS~Code}

\subsection*{3.1. Extensión de Python}
\begin{enumerate}
  \item Abrir VS~Code.
  \item En la barra lateral izquierda, pulsar en el icono de extensiones (cuadro con cuatro bloques) o usar \texttt{Ctrl+Shift+X}.
  \item Buscar \texttt{Python}.
  \item Instalar la extensión oficial de Microsoft (\emph{Python} --- publisher \emph{Microsoft}).
\end{enumerate}

\subsection*{3.2. Seleccionar el intérprete de Python}
\begin{enumerate}
  \item Abrir la Paleta de Comandos: \texttt{Ctrl+Shift+P} (Windows/Linux) o \texttt{Cmd+Shift+P} (macOS).
  \item Escribir \texttt{Python: Select Interpreter}.
  \item Elegir el Python instalado (normalmente aparecerá como \texttt{Python 3.x.x} en una ruta estándar).
\end{enumerate}

\section*{4. Primera prueba: ``Hola, mundo''}

\begin{enumerate}
  \item En VS~Code, crear una carpeta para la asignatura y abrirla (\textbf{File $\rightarrow$ Open Folder}).
  \item Crear un archivo nuevo llamado \texttt{hola.py}.
  \item Escribir:
\begin{verbatim}
print("Hola, mundo")
\end{verbatim}
  \item Guardar el archivo.
  \item Abrir la terminal integrada (\textbf{Terminal $\rightarrow$ New Terminal}) y ejecutar:
\begin{verbatim}
python hola.py
\end{verbatim}
  (o \texttt{python3 hola.py} en macOS/Linux).
  \item Debería aparecer el mensaje \texttt{Hola, mundo} en la terminal. Si es así, el entorno está listo.
\end{enumerate}

\vfill
\noindent\textbf{Nota:} es recomendable que el alumnado traiga estas herramientas ya instaladas a la primera sesión de prácticas.

\end{document}

